% Please do not change the document class
\documentclass{scrartcl}

% Please do not change these packages
\usepackage[hidelinks]{hyperref}
\usepackage[none]{hyphenat}
\usepackage{setspace}
\doublespace

% You may add additional packages here
\usepackage{amsmath}

% Please include a clear, concise, and descriptive title
\title{COMP110 - Research Journal - Computing}

% Please put your student number in the author field
\author{1507729}

\begin{document}

\maketitle

\section{When does a physical system compute? When does it matter?}
The paper as written by Gilbert, Johnson and Keerthi \cite{horsman2014does}. Supposes that by gleaming insight from other mathmatical research it is possible to answer questions in the field of computer science. They propose a framework for testing if a computer is in the process of computing, though the applications of this seems rather limited I say this as in my opinion we don't work with systems to know when they compute just to know what they are computing and the results. They propose that the benefit of their research is the a unified definition for if an entity is a computer and to challenge whether or not the user is using the computer or allowing the use to compute the answer themselves.

\section{Experimental Investigations of the Utility of Detailed Flowcharts in Programming}
The following paper by Shneiderman, Mayer, McKay and Heller \cite{shneiderman1977experimental}  seeks to challenge the computational norms suggest initially by Goldstein and von Neumann \cite{goldstein5planning} in 1947 about the use of Flowcharts and the actual benefits they produce. They start by outlining their research experiemnts before begining to harvest quantifiable data. 

\section{A Fast Procedure for Computing the Distance Between Complex Objects in Three-Dimensional Space}

\section{Go To Statement Considered Harmful}

\section{Conclusion}

\bibliographystyle{ieeetran}
\bibliography{references}

\end{document}
